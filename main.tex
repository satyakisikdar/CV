% Don't like 10pt? Try 11pt or 12pt
\documentclass[11pt]{article}

\usepackage{hyperref}
\usepackage{calc}
\usepackage{comment}

\makeatletter
\newlength{\bibhang}
\setlength{\bibhang}{1em} %1em}
\newlength{\bibsep}
 {\@listi \global\bibsep\itemsep \global\advance\bibsep by\parsep}
\newenvironment{bibsection}%
        {\begin{enumerate}{}{%
%        {\begin{list}{}{%
       \setlength{\leftmargin}{\bibhang}%
       \setlength{\itemindent}{-\leftmargin}%
       \setlength{\itemsep}{\bibsep}%
       \setlength{\parsep}{\z@}%
        \setlength{\partopsep}{0pt}%
        \setlength{\topsep}{0pt}}}
        {\end{enumerate}\vspace{-.6\baselineskip}}
%        {\end{list}\vspace{-.6\baselineskip}}
\makeatother

% Layout: Puts the section titles on left side of page
\reversemarginpar

%
%         PAPER SIZE, PAGE NUMBER, AND DOCUMENT LAYOUT NOTES:
%
% The next \usepackage line changes the layout for CV style section
% headings as marginal notes. It also sets up the paper size as either
% letter or A4. By default, letter was used. If A4 paper is desired,
% comment out the letterpaper lines and uncomment the a4paper lines.
%
% As you can see, the margin widths and section title widths can be
% easily adjusted.
%
% ALSO: Notice that the includefoot option can be commented OUT in order
% to put the PAGE NUMBER *IN* the bottom margin. This will make the
% effective text area larger.
%
% IF YOU WISH TO REMOVE THE ``of LASTPAGE'' next to each page number,
% see the note about the +LP and -LP lines below. Comment out the +LP
% and uncomment the -LP.
%
% IF YOU WISH TO REMOVE PAGE NUMBERS, be sure that the includefoot line
% is uncommented and ALSO uncomment the \pagestyle{empty} a few lines
% below.
%

%% Use these lines for letter-sized paper
\usepackage[paper=letterpaper,
            %includefoot, % Uncomment to put page number above margin
            marginparwidth=1.2in,     % Length of section titles
            marginparsep=.05in,       % Space between titles and text
            margin=0.75in,               % 1 inch margins
            includemp]{geometry}

%% Use these lines for A4-sized paper
%\usepackage[paper=a4paper,
%            %includefoot, % Uncomment to put page number above margin
%            marginparwidth=30.5mm,    % Length of section titles
%            marginparsep=1.5mm,       % Space between titles and text
%            margin=25mm,              % 25mm margins
%            includemp]{geometry}

%% More layout: Get rid of indenting throughout entire document
\setlength{\parindent}{0in}
\usepackage[T1]{fontenc}
\usepackage[shortlabels]{enumitem}

%% Reference the last page in the page number
%
% NOTE: comment the +LP line and uncomment the -LP line to have page
%       numbers without the ``of ##'' last page reference)
%
% NOTE: uncomment the \pagestyle{empty} line to get rid of all page
%       numbers (make sure includefoot is commented out above)
%
\usepackage{fancyhdr,lastpage}
\pagestyle{fancy}
\pagestyle{empty}      % Uncomment this to get rid of page numbers
\fancyhf{}\renewcommand{\headrulewidth}{0pt}
\fancyfootoffset{\marginparsep+\marginparwidth}
\newlength{\footpageshift}
\setlength{\footpageshift}
          {0.5\textwidth+0.5\marginparsep+0.5\marginparwidth-2in}
\lfoot{\hspace{\footpageshift}%
       \parbox{4in}{\, \hfill %
                    \arabic{page} of \protect\pageref*{LastPage} % +LP
%                    \arabic{page}                               % -LP
                    \hfill \,}}

% Finally, give us PDF bookmarks
\usepackage{color,hyperref}
\definecolor{darkblue}{rgb}{0.0,0.0,0.3}
\hypersetup{colorlinks,breaklinks,
            linkcolor=darkblue,urlcolor=darkblue,
            anchorcolor=darkblue,citecolor=darkblue}

%%%%%%%%%%%%%%%%%%%%%%%% End Document Setup %%%%%%%%%%%%%%%%%%%%%%%%%%%%


%%%%%%%%%%%%%%%%%%%%%%%%%%% Helper Commands %%%%%%%%%%%%%%%%%%%%%%%%%%%%

% The title (name) with a horizontal rule under it
% (optional argument typesets an object right-justified across from name
%  as well)
%
% Usage: \makeheading{name}
%        OR
%        \makeheading[right_object]{name}
%
% Place at top of document. It should be the first thing.
% If ``right_object'' is provided in the square-braced optional
% argument, it will be right justified on the same line as ``name'' at
% the top of the CV. For example:
%
%       \makeheading[\emph{Curriculum vitae}]{Your Name}
%
% will put an emphasized ``Curriculum vitae'' at the top of the document
% as a title. Likewise, a picture could be included:
%
%   \makeheading[\includegraphics[height=1.5in]{my_picutre}]{Your Name}
%
% the picture will be flush right across from the name.
\newcommand{\makeheading}[2][]%
        {\hspace*{-\marginparsep minus \marginparwidth}%
         \begin{minipage}[t]{\textwidth+\marginparwidth+\marginparsep}%
             {\large \bfseries #2 \hfill #1}\\[-0.15\baselineskip]%
                 \rule{\columnwidth}{1pt}%
         \end{minipage}}

% The section headings
%
% Usage: \section{section name}
\renewcommand{\section}[1]{\pagebreak[3]%
    \hyphenpenalty=10000%
    \vspace{1.3\baselineskip}%
    \phantomsection\addcontentsline{toc}{section}{#1}%
    \noindent\llap{\scshape\smash{\parbox[t]{\marginparwidth}{\raggedright #1}}}%
    \vspace{-\baselineskip}\par}

% An itemize-style list with lots of space between items
\newenvironment{outerlist}[1][\enskip\textbullet]%
        {\begin{itemize}[#1,leftmargin=*]}{\end{itemize}%
         \vspace{-.6\baselineskip}}

% An environment IDENTICAL to outerlist that has better pre-list spacing
% when used as the first thing in a \section
\newenvironment{lonelist}[1][\enskip\textbullet]%
        {\begin{list}{#1}{%
        \setlength{\partopsep}{0pt}%
        \setlength{\topsep}{0pt}}}
        {\end{list}\vspace{-.6\baselineskip}}

% An itemize-style list with little space between items
\newenvironment{innerlist}[1][\enskip]%
        {\begin{enumerate}[#1,leftmargin=*,parsep=0pt,itemsep=0pt,topsep=0pt,partopsep=0pt]}
        {\end{enumerate}}

% An environment IDENTICAL to innerlist that has better pre-list spacing
% when used as the first thing in a \section
\newenvironment{loneinnerlist}[1][\enskip\textbullet]%
        {\begin{itemize}[#1,leftmargin=*,parsep=0pt,itemsep=0pt,topsep=0pt,partopsep=0pt]}
        {\end{itemize}\vspace{-.6\baselineskip}}

% To add some paragraph space between lines.
% This also tells LaTeX to preferably break a page on one of these gaps
% if there is a needed pagebreak nearby.
\newcommand{\blankline}{\quad\pagebreak[3]}
\newcommand{\halfblankline}{\quad\vspace{-0.5\baselineskip}\pagebreak[3]}

% Uses hyperref to link DOI
\newcommand\doilink[1]{\href{http://dx.doi.org/#1}{#1}}
\newcommand\doi[1]{doi:\doilink{#1}}

% For \url{SOME_URL}, links SOME_URL to the url SOME_URL
\providecommand*\url[1]{\href{#1}{#1}}
% Same as above, but pretty-prints SOME_URL in teletype fixed-width font
\renewcommand*\url[1]{\href{#1}{\texttt{#1}}}

% For \email{ADDRESS}, links ADDRESS to the url mailto:ADDRESS
\providecommand*\email[1]{\href{mailto:#1}{#1}}
% Same as above, but pretty-prints ADDRESS in teletype fixed-width font
%\renewcommand*\email[1]{\href{mailto:#1}{\texttt{#1}}}

%\providecommand\BibTeX{{\rm B\kern-.05em{\sc i\kern-.025em b}\kern-.08em
%    T\kern-.1667em\lower.7ex\hbox{E}\kern-.125emX}}
%\providecommand\BibTeX{{\rm B\kern-.05em{\sc i\kern-.025em b}\kern-.08em
%    \TeX}}
\providecommand\BibTeX{{B\kern-.05em{\sc i\kern-.025em b}\kern-.08em
    \TeX}}
\providecommand\Matlab{\textsc{Matlab}}

%%%%%%%%%%%%%%%%%%%%%%%% End Helper Commands %%%%%%%%%%%%%%%%%%%%%%%%%%%

%%%%%%%%%%%%%%%%%%%%%%%%% Begin CV Document %%%%%%%%%%%%%%%%%%%%%%%%%%%%

\begin{document}
\makeheading{{\huge  Satyaki Sikdar} \hfill \scriptsize{\textmd{Updated: \today}}}

\section{Contact Information}

% NOTE: Mind where the & separators and \\ breaks are in the following
%       table.
%
% ALSO: \rcollength is the width of the right column of the table
%       (adjust it to your liking; default is 1.85in).
%
\newlength{\rcollength}\setlength{\rcollength}{1.4in}%
%
\begin{tabular}[t]{@{}p{\textwidth-1.5\rcollength}p{\rcollength}}
%\href{http://www.cse.osu.edu/}%
%     {Department of Computer Science and Engineering} & \\
%\href{http://www.osu.edu/}{The Ohio State University}
310 Stinson-Remick Hall   & \href{https://sites.nd.edu/ssikdar/}{https://sites.nd.edu/ssikdar/} \\
Notre Dame, IN 46556, USA     & \email{ssikdar@nd.edu}\\
\end{tabular}

%\section{Objective}

%Insert text here if you want to
%\begin{innerlist}
%\item More information and auxiliary documents can be found at\\\url{http://www.tedpavlic.com/facjobsearch/}
%\end{innerlist}

\section{Research Interests}

Network science, data mining, and machine learning

\section{Education}

\href{http://nd.edu}{\textbf{University of Notre Dame}},
Notre Dame, IN
\begin{outerlist}
\item[] Ph.D., 
        \href{http://cse.nd.edu}
             {Computer Science \& Engineering}, GPA: $3.93/4$
             \hfill $2017$ -- 
        \begin{innerlist}
        \item[] Adviser:
              \href{https://www3.nd.edu/~tweninge/}
                   {Tim Weninger, Ph.D.} 
        \end{innerlist}
\end{outerlist}

\halfblankline

\href{http://heritageit.edu/}{\textbf{Heritage Institute of Technology}, Kolkata, India}    
\begin{outerlist}
	\item[] B.Tech., 
        \href{http://heritageit.edu/CSE.aspx}
             {Computer Science \& Engineering}, GPA: $8.8 / 10$
            \hfill $2013$ -- $2017$
        \begin{innerlist}
        \item[] Thesis: \emph{Learning Models for Influence Maximization}
        \item[] Advisers: Partha Basuchowdhuri, Ph.D. and Subhashis Majumder, Ph.D.
        \end{innerlist}
\end{outerlist}

\section{Research Experience}

\textbf{Graduate Research Assistant} \hfill {Aug $2017$ --}
\begin{innerlist}

\item[] Department of Computer Science \& Engineering,\\
        University of Notre Dame\\
        Supervisor: Tim Weninger, Ph.D.
\end{innerlist}

\halfblankline

\textbf{Research Assistant} \hfill {Apr $2015$ -- Jul $2017$}
\begin{innerlist}

\item[] Department of Computer Science \& Engineering,\\
        Heritage Institute of Technology, Kolkata\\
        Supervisors: Partha Basuchowdhuri, Ph.D. and Subhashis Majumder, Ph.D.
\end{innerlist}

\section{Journal Publications}
\vspace{-.125in}
\begin{bibsection}
    \item[J$1$.] Basuchowdhuri, P., {\bf Sikdar, S.}, Nagarajan, V., Mishra, K., Gupta, K., and Majumder, S. \emph{Fast Detection of Community Structures using Graph Traversal in Social Networks}, 2017. Knowledge and Information Systems (KAIS). \href{https://arxiv.org/abs/1707.04459}{arXiv:1707.04459}.
\end{bibsection}

\section{Conference Publications}
\vspace{-.125in}
\begin{bibsection}
    \item[C$2$.] Pennycuff, C., \textbf{Sikdar, S.}, Vajiac, C., Chiang, D., and Weninger, T., 2018, June. \emph{Synchronous Hyperedge Replacement Graph Grammars}. International Conference on Graph Transformation (ICGT), 2018.   

	\item[C$1$.] Basuchowdhuri, P., \textbf{Sikdar, S.}, Shreshtha, S., and Majumder, S., 2016, March. \emph{Detecting Community Structures in Social Networks by Graph Sparsification}. In Proceedings of the $3^{rd}$ IKDD Conference on Data Science, 2016 (p. 5). ACM.	
\end{bibsection}

\section{Papers Under Review}
\vspace{-.125in}
\begin{bibsection}
    \item[JR$1$.] \textbf{Sikdar, S.}, Hibshman J., and Weninger, T. \textit{Modelling Graphs with Vertex Rep-lacement Grammars}, under  review in the Journal Track of ECML-PKDD, 2019. 
\end{bibsection}

% Add a little space to nudge next ``Conference Publications'' marginpar
% down to make room for tall ``Submitted Journal Publications''
% marginpar. If there are enough submitted journal publications, this
% space will not be needed (and should be removed).
%\vspace{0.1in}


\section{Awards}
\textbf{Travel Awards}
\begin{innerlist}
\item[] ACM IKDD Conference on Data Science $2016$ \hfill Mar $2016$
\end{innerlist}

\halfblankline

\textbf{Student Awards} --- Heritage Institute of Technology, Kolkata
\begin{innerlist}
\item[] Best Student Award for Academic Excellence \hfill Jul $2017$
\end{innerlist}

\halfblankline

\section{Projects}
\textbf{Modelling graphs with Vertex Replacement Grammars} \hfill $2018$ -- 
\begin{loneinnerlist}
    \item[] Created a graph framework and related algorithms that extracts structural features from a given graph and uses that to generate a family of topologically similar graphs. Tools used: Python with the NetworkX library. \href{https://github.com/satyakisikdar/VRG}{(Github repository)}
\end{loneinnerlist}

\halfblankline

\textbf{Synchronous Hyperedge Replacement Graph Grammars} \hfill $2017$ -- $2018$
\begin{loneinnerlist}
    \item[] Created a graph framework and related algorithms that generalizes language translation for modeling temporal graphs. Tools used: Python with the NetworkX library. \href{https://github.com/tweninger/PSHRG}{(Github repository)}
\end{loneinnerlist}

\blankline

\textbf{Learning Models for Influence Maximization} \hfill $2016$ -- $2017$

\begin{loneinnerlist}
    \item[] Designed a recommender system for recommending restaurants to users based on the topology of the underlying bipartite network of users and restaurants. The data was crawled from a popular restaurant review site. Tools used: Python with Selenium, BeautifulSoup, Pandas, numpy, NetworkX, and scipy libraries. 
\end{loneinnerlist}

\vspace*{0.5cm}    
\textbf{Community Detection in Social Networks} \hfill $2015$ -- $2017$
    \begin{innerlist}
        \item[] \textit{Using Graph Traversal Techniques}
        \begin{innerlist}
            \item[] Worked on implementation and testing of a novel community detection algorithm that uses a mix of breadth-first and depth-first traversals for fast unveiling of communities. Tools used: C, C++, and shell scripts.  \href{https://github.com/sna-lincom/LINCOM}{(Github repository)}
        \end{innerlist}   
        \vspace{0.2cm}
        \item[] \textit{Using Graph Sparsification} 
        \begin{innerlist}
            \item[] Worked on design, implementation, and testing of a fast community detection algorithm that uses a geometric $t$-spanner to identify the edges with high edge betweeenness and thus unraveling the community structure. Tools used: C++ with Boost Graph Library, Python with NetworkX library. \href{https://github.com/satyakisikdar/spanner-comm-detection}{(Github repository)}
        \end{innerlist}
    \end{innerlist}
% \end{outerlist}

\halfblankline

\section{Presentations}

\textbf{Paper Presentations}
\begin{innerlist}
    \item[] \href{https://goo.gl/pDRGwY}{\textit{Synchronous Hyperedge Replacement Graph Grammars}}, International \\  Conference on Graph Transformations, $2018$, Toulouse, France \hfill Jun $2018$
    \item[] \href{https://bit.ly/2IESM33}{\textit{Synchronous Hyperedge Replacement Graph Grammars}}, Midwest \\ Speech \& Language Days, $2018$, Notre Dame, IN, US \hfill May $2018$
	\item[] \href{http://bit.ly/2fMiEID}{\textit{Detecting Community Structures in Social Networks by Graph \\ Sparsification}}, ACM IKDD Conference on Data Science, $2016$ \hfill Mar $2016$
\end{innerlist} 

\halfblankline

\textbf{SIGKDD, ACM Student Chapter, Heritage Institute of Technology}
\begin{innerlist}
\item \href{http://bit.ly/2oKfjSn}{Introduction to Support Vector Machines} \hfill Apr $2017$
\item \href{http://bit.ly/2lZJhNL}{Density Based Spatial Clustering of Applications with Noise} \hfill Feb $2017$
\item \href{http://bit.ly/2eNQmRe}{Introduction to Decision Trees}\hfill Nov $2016$
\item \href{http://bit.ly/2f7O3UJ}{A Friendly Introduction to Random Networks}\hfill Feb $2016$

\end{innerlist}

\halfblankline

\textbf{Invited Talks at Vidyasagar College, Kolkata}
\begin{innerlist}
\item \href{http://bit.ly/2g47S16}{An Introduction to Community Detection in Social Networks} \hfill Jan $2016$
\item Massive Open Online Courses \hfill Sep $2014$
\end{innerlist}


\section{Teaching Experience}

%\textbf{Teaching Assistant} \hfill {Springs 2011--12}
\textbf{Graduate Teaching Assistant} \hfill {Spring $2018$}
\begin{innerlist}
\item[] \emph{CSE $30151$ - Theory of Computing}
    \begin{innerlist}
        \item[] Instructor: David Chiang, Ph.D
        
        \item[] Department of Computer Science \& Engineering,
        
        \item[] University of Notre Dame
        
        \item[] Graded assignments, exams, held office hours, and designed a \href{https://www3.nd.edu/~dchiang/teaching/theory/2018/www/tikz_tutorial.pdf}{tutorial} on drawing \\ finite state machine with \texttt{TikZ} library
    \end{innerlist}
\end{innerlist}

% \halfblankline
\newpage
\textbf{Graduate Teaching Assistant} \hfill {Fall $2017$}
\begin{innerlist}
\item[] \emph{CSE $30151$ - Theory of Computing}
    \begin{innerlist}
        \item[] Instructor: Peter M. Kogge, Ph.D
        
        \item[] Department of Computer Science \& Engineering,
        
        \item[] University of Notre Dame
        
        \item[] Graded assignments, exams, held office hours, and designed a \href{https://www3.nd.edu/~kogge/courses/cse30151-fa17/Public/other/tikz_tutorial.pdf}{tutorial} on drawing \\ finite state machine with \texttt{TikZ} library
    \end{innerlist}
\end{innerlist}

\halfblankline

\textbf{Lecture Series}

\begin{innerlist}
	\item[] \emph{Introduction to Programming in Python} \hfill Fall $2016, 2015$
	\begin{innerlist}
	    \item[] A 15 hour introductory course on programming in Python
	\end{innerlist}
\end{innerlist}

\halfblankline

\textbf{Workshops}
\begin{innerlist}
	\item[] \emph{Introduction to Programming in Python} \hfill Apr $2016, 2015$
	\begin{innerlist}
	    \item[] A two-day introductory hands-on workshop on programming in Python
	\end{innerlist}
\end{innerlist}


\section{Service}
\emph{Subreviewer}, The Web Conference $2019$ \hfill {$2018$}
\vspace*{0.1cm}

\emph{Reviewer}, Data Mining and Knowledge Discovery (DMKD) \hfill {$2018$}
\vspace*{0.1cm}

\emph{Reviewer}, International Journal of Cooperative Information Systems (IJCIS) \hfill {$2018$}
\vspace*{0.1cm}

\emph{Subreviewer}, AAAI $2018$ \hfill {$2017$}
\vspace*{0.1cm}

\emph{Chair}, ACM Student Chapter, Heritage Institute of Technology  \hfill $2016$ -- $2017$
\vspace*{0.1cm}

\emph{Vice Chair}, ACM Student Chapter, Heritage Institute of Technology \hfill $2015$ -- $2016$

\vspace*{0.1cm}

\emph{Secretary}, ACM Student Chapter, Heritage Institute of Technology \hfill $2014$ -- $2015$

\vspace*{0.1cm}

\emph{Student Member}, ACM \hfill{$2013$ --}

\section{Computer Skills}
%
\textbf{Advanced}
\begin{innerlist}
	\item[] C, Python with NetworkX, matplotlib, Selenium, and BeautifulSoup libraries
\end{innerlist}

\halfblankline

\textbf{Intermediate} 
\begin{innerlist}
	\item[] C++, Boost Graph Library, \LaTeX{} with Beamer class, Pandas, SQL, Shell scripts
\end{innerlist}

\halfblankline

\textbf{Basic}
\begin{innerlist}
\item[] Java, C\#, WEKA, PySpark
\end{innerlist}


\section{References}

\textbf{Tim Weninger} \hfill {Phone: +$1$-$574$-$631$-$6770$}
\begin{innerlist}
\item[] Assistant Professor \\
Department of Computer Science \& Engineering,\\
University of Notre Dame \\
E-mail: \email{tweninger@nd.edu}
\end{innerlist}

\halfblankline

\textbf{Peter M. Kogge} \hfill {Phone: +$1$-$574$-$631$-$6763$}
\begin{innerlist}
\item[] Ted H. McCourtney Professor of Computer Science \& Engineering \\
IBM Fellow, IEEE Fellow \\
Department of Computer Science \& Engineering,\\
University of Notre Dame \\
E-mail: \email{kogge@nd.edu}
\end{innerlist}



\end{document}

%%%%%%%%%%%%%%%%%%%%%%%%%% End CV Document %%%%%%%%%%%%%%%%%%%%%%%%%%%%%

%----------------------------------------------------------------------%
% The following is copyright and licensing information for
% redistribution of this LaTeX source code; it also includes a liability
% statement. If this source code is not being redistributed to others,
% it may be omitted. It has no effect on the function of the above code.
%----------------------------------------------------------------------%
% Copyright (c) 2007, 2008, 2009, 2010, 2011 by Theodore P. Pavlic
%
% Unless otherwise expressly stated, this work is licensed under the
% Creative Commons Attribution-Noncommercial 3.0 United States License. To
% view a copy of this license, visit
% http://creativecommons.org/licenses/by-nc/3.0/us/ or send a letter to
% Creative Commons, 171 Second Street, Suite 300, San Francisco,
% California, 94105, USA.
%
% THE SOFTWARE IS PROVIDED "AS IS", WITHOUT WARRANTY OF ANY KIND, EXPRESS
% OR IMPLIED, INCLUDING BUT NOT LIMITED TO THE WARRANTIES OF
% MERCHANTABILITY, FITNESS FOR A PARTICULAR PURPOSE AND NONINFRINGEMENT.
% IN NO EVENT SHALL THE AUTHORS OR COPYRIGHT HOLDERS BE LIABLE FOR ANY
% CLAIM, DAMAGES OR OTHER LIABILITY, WHETHER IN AN ACTION OF CONTRACT,
% TORT OR OTHERWISE, ARISING FROM, OUT OF OR IN CONNECTION WITH THE
% SOFTWARE OR THE USE OR OTHER DEALINGS IN THE SOFTWARE.
%----------------------------------------------------------------------%
